%# -*- coding: utf-8 -*-
\ifx\epTeXinputencoding\undefined\else % defined in e-pTeX (> TL2016)
  \epTeXinputencoding utf8    % ensure utf-8 encoding for platex
\fi

\documentclass[a4paper]{jsarticle}
\usepackage{doc}
\usepackage{pxstfloats}
\GetFileInfo{pxstfloats.sty}
\def\Lpack#1{\textsf{#1}}
\def\Lopt#1{\texttt{#1}}
\title{Package \Lpack{pxstfloats} \fileversion}
\author{Hironobu Yamashita}
\date{\filedate}
\begin{document}

\maketitle

\Lpack{pxstfloats}パッケージは、Sigitas Tolu\v{s}氏による
\Lpack{stfloats}パッケージ(\Lpack{sttools}バンドル)をp\LaTeX{}でも
使えるようにするパッチを提供します。

\section{\Lpack{stfloats}パッケージの挙動について}

\Lpack{stfloats}パッケージは脚注とボトムフロートの位置を変更する
2つの命令を定義しています。説明文書には以下のように書かれています:
\begin{quote}
\verb+\fnbelowfloat+:
  Puts footnotes below the bottom floats.\par
\verb+\fnunderfloat+ (default and standard):
  Puts footnotes under the bottom floats.
\end{quote}
どちらも脚注をボトムフロートより下に置くという意味ですので、この説明は
奇妙です。しかも、\LaTeXe{}カーネルの標準では脚注はボトムフロートの
\emph{上}に置かれますし、\Lpack{stfloats}パッケージ自身を\LaTeX{}上で
読み込んだ場合のデフォルトもそれと同じだからです。これらの挙動に従えば、
\verb+\fnunderfloat+は本来\verb+\fnabovefloat+と呼ばれるべきものです。
このことは2007年の\verb+comp.text.tex+での議論にも上っています:
\begin{quote}\begin{verbatim}
https://groups.google.com/forum/#!msg/comp.text.tex/PA1IBKczm4A/wXlgCe_2ukIJ
\end{verbatim}\end{quote}

さらにややこしいことに、日本のp\LaTeXe{}カーネルの標準は
\LaTeXe{}カーネルと異なり、最初に公開された時から一貫して脚注は
ボトムフロートの\emph{下}に置かれるように設計されています\footnote{%
この設計はp\LaTeXe{}の原著者であるアスキー社によって、日本の組版で
一般的な方法に従うための意図的なものです。2016年以降のコミュニティ版
p\LaTeX{}や、派生物であるup\LaTeXe{}も同じ挙動になっています。}。
しかし\Lpack{stfloats}の実装上の問題により、\verb+\fnbelowfloat+と
\verb+\fnunderfloat+のどちらが有効かに関わらず、脚注は常に
ボトムフロートの\emph{下}に現れてしまいます\footnote{%
\Lpack{stfloats}はカーネルが「脚注をボトムフロートの上に置く」と
いう定義であると仮定し、無条件にその定義を保存します。}。
つまり、脚注をボトムフロートの上に出すことが出来ません。

\section{\Lpack{pxstfloats}パッケージの機能}

上の状況に対処するため、\Lpack{pxstfloats}パッケージは以下の機能を
提供します:
\begin{quote}
\verb+\fnbelowfloat+:
  脚注をボトムフロートの下に置きます。\par
\verb+\fnabovefloat+:
  脚注をボトムフロートの上に置きます。\par
\verb+\fnunderfloat+ (default):
  現在使われているカーネル(\LaTeX{}またはp\LaTeX{})のまま
  変更しません。
\end{quote}
\verb+\fnbelowfloat+の挙動は元の\Lpack{stfloats}と同じで、文字どおりの
意味を持ちます。\verb+\fnunderfloat+の挙動も\LaTeXe{}とp\LaTeXe{}の
どちらのカーネルでも従来の挙動と同じになっています(文字どおりの
意味と異なるところまで同じです)。新設した\verb+\fnabovefloat+は
文字どおりの意味を持ち、\LaTeXe{}でもp\LaTeXe{}でも正常に機能します。

\section{補足事項}

Hiroshi Nakashima氏による\Lpack{fnpos}パッケージ(yafootバンドル)も、
\Lpack{pxstfloats}と同様の機能を提供しています。\LaTeX{}やp\LaTeX{}の
カーネルおよび\Lpack{pxstfloats}パッケージの挙動は、
\Lpack{fnpos}パッケージの用語に当てはめると次のようになります:
\begin{itemize}
\item \LaTeX\ kernel
      = \verb+\makeFNmid+ + \verb+\makeFNabove+
\item p\LaTeX\ kernel
      = \verb+\makeFNbottom+ + \verb+\makeFNbelow+
\item \verb+\fnbelowfloat+
      = \verb+\makeFNmid+ + \verb+\makeFNbelow+
\item \verb+\fnabovefloat+
      = \verb+\makeFNmid+ + \verb+\makeFNabove+
\end{itemize}

\end{document}
