%# -*- ascii characters only -*-
\documentclass[a4paper]{article}
\usepackage{doc}
\usepackage{pxstfloats}
\GetFileInfo{pxstfloats.sty}
\def\Lpack#1{\textsf{#1}}
\def\Lopt#1{\texttt{#1}}
\title{Package \Lpack{pxstfloats} \fileversion}
\author{Hironobu Yamashita}
\date{\filedate}
\begin{document}

\maketitle

The package \Lpack{pxstfloats} provides p\LaTeX\ support for
\Lpack{stfloats} package written by Sigitas Tolu\v{s}is, included in
\Lpack{sttools} bundle.

\section{The behavior of \Lpack{stfloats} package}

The documentation of \Lpack{stfloats} package describes two commands
as follows:
\begin{quote}
\verb+\fnbelowfloat+:
  Puts footnotes below the bottom floats.\par
\verb+\fnunderfloat+ (default and standard):
  Puts footnotes under the bottom floats.
\end{quote}
However, this statement is strange enough, as the standard behavior
of \LaTeXe\ kernel is to put footnotes \emph{above} the bottom
floats. Moreover, the same applies for the default behavior of
\Lpack{stfloats} itself, when used under the standard \LaTeXe.
According to these behaviors, \verb+\fnunderfloat+ really should be
called \verb+\fnabovefloat+. See an old discussion in 2007 at
\verb+comp.text.tex+:
\begin{quote}\begin{verbatim}
https://groups.google.com/forum/#!msg/comp.text.tex/PA1IBKczm4A/wXlgCe_2ukIJ
\end{verbatim}\end{quote}

More confusingly, the ``standard'' behavior of Japanese p\LaTeXe\
kernel is different from that of \LaTeXe\ kernel; the footnotes are
put \emph{below} the bottom floats, since the first release of
p\LaTeXe.\footnote{The decision was made by the original author,
ASCII Corporation, based on the common way of Japanese traditional
publishing. The Unicode version of it, called up\LaTeXe, also has
the same behavior.} Due to the implementation of \Lpack{stfloats}
package, the footnotes always appear \emph{below} the bottom floats
when it is used under p\LaTeXe, no matter whether \verb+\fnbelowfloat+
or \verb+\fnunderfloat+ is effective.

\section{Introduction to \Lpack{pxstfloats} package}

Considering these situations, the package \Lpack{pxstfloats} provides
the following features:
\begin{quote}
\verb+\fnbelowfloat+:
  Puts footnotes below the bottom floats.\par
\verb+\fnabovefloat+:
  Puts footnotes above the bottom floats.\par
\verb+\fnunderfloat+ (default):
  Restores the default behavior of the kernel (\LaTeX\ or
  p\LaTeX) which is currently used.
\end{quote}
The behavior of \verb+\fnbelowfloat+ is the same as that of the
original \Lpack{stfloats}, which agrees with its literal meaning.
The behavior of \verb+\fnunderfloat+ is kept unchanged for backward
compatibility with all kernels (both \LaTeXe\ and p\LaTeXe), though
it may not agree with the literal meaning of its name. The new macro
\verb+\fnabovefloat+ is added, which follows its literal meaning.

\section{Additional notes}

The package \Lpack{fnpos} in yafoot bundle (by Hiroshi Nakashima)
also provides a similar function. The behavior of
\LaTeX/p\LaTeX\ kernels and the effect of \Lpack{pxstfloats}
can be explained in \Lpack{fnpos} syntax as follows:
\begin{itemize}
\item \LaTeX\ kernel
      = \verb+\makeFNmid+ + \verb+\makeFNabove+
\item p\LaTeX\ kernel
      = \verb+\makeFNbottom+ + \verb+\makeFNbelow+
\item \verb+\fnbelowfloat+
      = \verb+\makeFNmid+ + \verb+\makeFNbelow+
\item \verb+\fnabovefloat+
      = \verb+\makeFNmid+ + \verb+\makeFNabove+
\end{itemize}

\end{document}
